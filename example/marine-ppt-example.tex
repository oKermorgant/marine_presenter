\documentclass{ecnbeamer}
\usepackage{multicol}

\title{3D presentations within Coral}
\author{Olivier Kermorgant}
\date{Today}

\beamertemplatenavigationsymbolsempty

\graphicspath{{fig/}}

\videoOFF

% TODO

\begin{document}

\MakeTitleNoFoot
\FootPage

\begin{frame}{Overview}
 
\begin{itemize}
\item ~ \vfill
\begin{itemize}
 \item Prerequisites\vfill
 \item Build a PDF\vfill
 \item Write the YAML file\vfill
 \item Generate simulation files \vfill
 \item Run presentation
\end{itemize}
\end{itemize}
\placeImage{shark}{2cm}{6cm}{.5\linewidth}
\end{frame}

\begin{frame}{Prerequisites}
\begin{itemize}[<+->]
 \item \okttt{marine_presenter} requires:\vfill
 \begin{itemize}
  \item ROS 2 (hence Ubuntu or similar distribution)\vfill
  \item Coral
  \begin{itemize}
   \item \okttt{https://github.com/oKermorgant/coral}
  \end{itemize}\vfill
    \item \okttt{slider_publisher}
  \begin{itemize}
   \item \okttt{apt install ros-foxy-slider-publisher}
  \end{itemize}\vfill
  \item PDF tools:
  \begin{itemize}
   \item \okttt{pdfinfo}
   \item \okttt{pdftotext} to get the titles of the slides
   \item \okttt{pdftoppm} to convert the pages to images
  \end{itemize}
 \end{itemize}
\end{itemize} 
\end{frame}


\begin{frame}{Building the PDF}
\begin{itemize}
 \item PDF can come from:
 \begin{itemize}
  \item Beamer / LaTeX\vfill
  \item Powerpoint or LibreOffice
 \end{itemize}\vfill
 \item Videos should not be included in the PDF\vfill
\end{itemize} 
\end{frame}

\begin{frame}{Configuring the presentation}
 \begin{itemize}
  \item Configuration is done through a YAML file with the same name as the PDF\vfill
  \item This file contains:\vfill
  \begin{itemize}[<+->]
   \item General informations (default slide pose, slide scale)\vfill
   \item Per-slide info: pose or associated videos\vfill
   \item Additional objects info: relative to slide or global to the environment
  \end{itemize}
 \end{itemize}
\end{frame}

\begin{frame}{Slide configuration}
 \begin{itemize}
  \item The YAML key for a given slide should be:
  \begin{itemize}
   \item the slide number
   \item the slide title (tested with LaTeX)
  \end{itemize}\vfill
  \item The \okttt{pose} sub-key is given in $(t_x, t_y, t_z, \text{roll}, \text{pitch}, \text{yaw})$
  \begin{itemize}
   \item Unless specified, the next slides use the latest given \okttt{pose}
  \end{itemize}\vfill
  \item The \okttt{video} sub-key indicates absolute or relative path to video file
  \begin{itemize}
   \item Sub-key can be \okttt{video1}, \okttt{video2}, etc. in case of multiple slides with the same title
   \item Videos can be played from the presentation by using the \okttt{pause} button of the remote
   \item VLC is used in command line to play the videos
  \end{itemize}
 \end{itemize}
\end{frame}

\begin{frame}{Object configuration}
\begin{itemize}
 \item Additional objects are given below the \okttt{objects} key
 \begin{itemize}
  \item For each object, the key is the name of the URDF file (without extension)
  \item The \okttt{center} sub-key is the object pose
  \item If a \okttt{slide} sub-key is given then the pose is relative to this slide\vfill
  \item Moving along an ellipse
  \begin{itemize}
   \item If subkeys \okttt{rx, ry, t} are given then the object will follow an ellipse centered on the given frame
   \item \okttt{rx} and \okttt{ry} are the width and depth 
   \item \okttt{t} is the time (in seconds) to do a full orbit
   \item If \okttt{roll} is given then the object has a roll proportional to its linear velocity
  \end{itemize}\vfill
  \item Moving a joint
  \begin{itemize}
   \item If the object has revolute joints, then the \okttt{joints} sub-key can be given as a list of times to perform a full circle on each of the joints 
  \end{itemize}\vfill
  \item All times can be negative to rotate in the other direction
 \end{itemize}
\end{itemize}
\end{frame}

\begin{frame}{Generating and launching}
 \begin{itemize}
  \item Generating the launch file is done through:\\
  \okttt{ros2 run marine_presenter generate.py <path to PDF file>}\vfill
  \item It will create a folder with the same name as the PDF\vfill
  \item Launching is performed with the \okttt{presentation_launch.py} from this folder\\
  \okttt{ros2 launch presentation_launch.py}
  
 \end{itemize}

\end{frame}








\end{document}
